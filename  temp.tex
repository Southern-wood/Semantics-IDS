\section{Background}
\label{sec:background}
\noindent{\textbf{ICS Attacks and Anomaly Detection.}} The ICS attack is to interfere with the normal operation of the industrial process. The typical ICS attack method is taking control of some devices in the ICS and injecting false data or commands that disobey the physical logic of the industrial process. Then, the ICS will react improperly, resulting in detrimental consequences such as industrial efficiency reduction or physical devices damages. Therefore, timely and accurate detection of ICS attacks is essential to ensure the security and safety of industrial processes. To detect the abnormal industrial process data caused by ICS attacks, most ICS anomaly detection methods learn the patterns of normal industrial process data based on learning-based approaches, such as statistical algorithms, data mining algorithms, machine learning and deep learning techniques, and utilize these patterns to distinguish between normal and abnormal real-time data. However, these methods often require high-quality data that precisely reflects the true physical state of the industrial process. Either data noise or tampering with the abnormal data to conceal attacks will degrade their detection performance.

\noindent{\textbf{Industrial Process Data Noise.}} 
The industrial process data is measured and reported by the physical components in ICS, such as sensors and actuators.
%and indicates the physical state of the industrial processes. 
Due to the imperfections of the physical components in real industrial environments, the collected industrial process data often contain non-negligible noise of varied types. For instance, the sensors' measurement errors usually lead to the gaussian noise~\cite{ahmed2018noise,ahmed2018noiseprint,lee2016deep}. Some aged physical components may become malfunctional or unsynchronized with other devices, making their process data being missing~\cite{li2022remaining,purohit2019mimii} or being delayed~\cite{el2023localizing,manousakis2018hybrid,zhang2018multiple}. Besides, the ICS may suffer undetected attacks, resulting in the normal industrial process data collected for training mixed with unidentified abnormal data. Overall, the noise in the industrial process data is of varied or even mixed types, and the type and the magnitude of noise is unknown in prior. Thus, it is impossible to filter out the data noise by simple data cleaning strategies.

\noindent{\textbf{ICS Attack Concealment.}} The ICS usually lacks strong security mechanism~\cite{green2017significance} due to its limited resources. As a result, the experienced attacker may hijack the communication channel between the physical components and the SCADA and tamper with the abnormal data, to conceal the attack activities.
%Due to the limited system resources, ICS usually lacks strict security verification mechanisms~\cite{green2017significance}. Thus, when attacking an ICS, the experienced attacker may apply the concealment attack strategy that hijacks the network communications and tampers with the process data reported to SCADA to be more similar to normal ones, to conceal their attack activities.
%may able to hijack the network communications and tamper with the process data reported to SCADA to be more simiar to normal data.
%This concealment attack strategy refers to modifying the abnormal process data reported to SCADA to look normal when launching the attacks that disrupt the industrial process.
A common concealment method is to modify the abnormal data to be the same as the previously collected normal data~\cite{ahmed2016limitations,mo2009secure,taormina2018deep,taormina2018battle}. Also, the attacker can block the communication channel during the attack to prevent physical components from reporting the abnormal data~\cite{krotofil2014cps}. In addition, the attacker can build a deep learning model that learns the intrinsic characteristics of normal data to transform abnormal data to be similar to normal~\cite{erba2020constrained}. When adopting the attack concealment methods, the characteristics of abnormal data is reduced and it cannot reflect the true abnormal physical state, hindering us from detecting the ICS attacks.

%When adopting these concealment attack strategies, the process data obtained by the anomaly detection model from SCADA cannot reflect the real abnormal physical state of the industrial process, increasing the difficulty of detecting anomalies.

%An ICS usually consists of many sensors, actuators, Programmable logic controllers (PLCs), and SCADA~\cite{stouffer2011guide}. The sensors continuously measure the physical data (in numerical values) that reflect the physical state of the industrial process and the actuators execute the physical operation commands received from the PLCs, where the operations are decided based on the measured sensor data and pre-defined process logic. The actuators utilize stateful data to represent their current physical operations. The industrial process data, i.e., the sensors' numerical data and the actuators' stateful data, are reported to the SCADA to support data analysis tasks, such as fault diagnosis~\cite{precup2015overview}, energy monitoring~\cite{figueiredo2012scada}, and anomaly detection~\cite{yang2006anomaly}. 

%The process data needs to reflect the physical state of the industrial process accurately, otherwise it may lead to incorrect data analysis results. 


%However, in a real ICS, many factors will make the collected process data noisy. For example, all physical components cannot function perfectly. Some of them may have data measurement errors, and the errors may increase gradually over time. Besides, the data reporting time of different physical components may not be synchronized, resulting in process data at different times, which are relevant to different physical states, being mistaken for simultaneous. Further, existing ICS anomaly detection methods require collecting some normal process data for training. But the ICS may suffer undetected attacks during the data collection, leading to the normal process data mixed with abnormal data.

%When accessing the ICS, numerous attack techniques can be applied to interfere with its industrial process. For instance, the attacker can exploit the vulnerabilities in ICS to inject false data and commands or infect the PLC with malware to disrupt its control logic. The goal of ICS anomaly detection is to detect abnormal industrial process data incurred by attacks. Traditional ICS anomaly detection techniques often require domain experts to directly model the physical logic of the industrial process, e.g., by defining linear or non-linear equations, which are laborious in realistic industrial environments and ill-suited to complicated industrial processes. The process data-based methods overcome this limitation by utilizing learning-based algorithms such as statistical methods, data mining, machine learning, and deep learning. 

%Due to the limited system resources, ICS usually lacks strict security verification mechanisms~\cite{green2017significance}. Thus, when attacking an ICS, the experienced attacker may apply the concealment attack strategy that hijacks the network communications and tampers with the process data reported to SCADA to be more similar to normal ones, to conceal their attack activities.
%may able to hijack the network communications and tamper with the process data reported to SCADA to be more simiar to normal data.
%This concealment attack strategy refers to modifying the abnormal process data reported to SCADA to look normal when launching the attacks that disrupt the industrial process.
%Typical concealment attack strategies include recording the normal process data before attacking and replaying it during the attack~\cite{ahmed2016limitations,mo2009secure,taormina2018deep,taormina2018battle}, blocking the latest process data report~\cite{krotofil2014cps}, and transforming the abnormal data into normal via learning-based algorithms~\cite{erba2020constrained}. When adopting these concealment attack strategies, the process data obtained by the anomaly detection model from SCADA cannot reflect the real abnormal physical state of the industrial process, increasing the difficulty of detecting anomalies.

\section{Problem Statement}
\label{sec:problem statement}
This paper aims to develop an anomaly detection system that can identify the abnormal industrial process data caused by the attacks targeting the ICS. The same as existing works~\cite{aoudi2018truth,zhang2023unsupervised,feng2019systematic,fungattributions,feng2021time,alsaedi2022usmd,kravchik2021efficient,erba2022assessing,tuli2022tranad,abdelaty2020aads,abdelaty2021daics}, the physical components in the ICS continuously measure and report the industrial process data to the SCADA, from which our system obtains the data for anomaly detection. Our system first collects some normal industrial process data for training and then be deployed to test whether the real-time collected testing data is abnormal. 

However, the industrial process data collected from real industrial environments is often of low-quality, i.e., can not precisely reflect the real physical state of the industrial process. The low-quality industrial process data is incurred by two factors: the imperfections of the physical components make the data contain various types of noises, and the attacker tamper with the abnormal data reported by the physical components to conceal the attack. Thus, the goal of our system is to accurately detect anomalies from low-quality industrial process data.

%Similar to existing works~\cite{aoudi2018truth,zhang2023unsupervised,feng2019systematic,fungattributions,feng2021time,alsaedi2022usmd,kravchik2021efficient,erba2022assessing,tuli2022tranad,abdelaty2020aads,abdelaty2021daics}, our system passively collects the industrial process data of the ICS from the SCADA, including numerical sensor data and stateful actuation data. To deploy the system, the administrator of the ICS needs to collect some normal process data for training.Then, the system can detect whether the real-time process data (i.e., testing data) is abnormal. Due to the imperfections of the operations of the physical components, both the training and testing industrial process data obtained by the system are noisy. In addition, when perform attacking, the attacker will apply the concealment attack strategy that modifies the abnormal process data to be similar to the normal data to avoid detection. Note that we assume that the attacker cannot interfere with the operation of the detection system nor know its details like the model architecutre, i.e., the attacker can only modify the abnormal process data in the black box manner.

%Formally let $\mathbf{X}_t$ be the true industrial process data of an ICS at time $t$ and $y_t \in \left \{ 0,1 \right \}$ be the label of $\mathbf{X}_t$, where $0$ or $1$ represents normal or abnormal, respectively. $\mathbf{X}_t$ consists of $N$ numerical data of sensors and $M$ stateful data of actuations, i.e.., $\mathbf{X}_t = (\mathbf{x}_1,...,\mathbf{x}_N,\mathrm{x}_1,...,\mathrm{x}_M)$, where $\mathbf{x}_i$ and $\mathrm{x}_j$ is the numerical value of the $i$-th sensor data and the stateful data of the $j$-th actuation data, respectively. 

%let $\left ( x_i,y_i \right )$ be the pair of an encrypted %traffic 
%sample $x_i$ (\eg a flow or a session between two hosts) and its true label $y_i \in \left \{ 0,1 \right \}$, where $0$ or $1$ represents a normal or a malicious one, respectively. The inputs of a detection system are a low-quality training set $\widetilde{D}_{train}= \left \{ \left ( x_i,\widetilde{y}_i \right )  \right \}_{i=1}^{N} $ and a testing set $D_{test}=\left \{ x^{test} \right \} $, where $N \ll \left | D_{test} \right | $ and $\widetilde{y}_i$ is a \emph{noisy} label that may be inconsistent with $y_i$. Our goal is to accurately infer the label of $x^{test}$ by using $\widetilde{D}_{train}$.

%deviates from the real physical state
Our system overcomes the limitations of existing ICS anomaly detection systems~\cite{aoudi2018truth,zhang2023unsupervised,feng2019systematic,fungattributions,feng2021time,alsaedi2022usmd,kravchik2021efficient,erba2022assessing,tuli2022tranad,abdelaty2020aads,abdelaty2021daics}. In particular, we consider two problems that lead to low-quality industrial process data, i.e., the data noise and the attack concealment, whereas prior arts only try to solve one problem. This is more realistic, yet more challenging for robust ICS anomaly detection. Note that we assume the attacker can only modify part of the industrial process data, which is the same as existing studies. Hijacking the communication channels of all physical components and modifying all data is a too powerful and unrealistic threat model, and is better be addressed by utilizing the information in other domains like side channel to perform anomaly detection. Further, our work can handle the low-quality data having varied or even mixed types of noise, while some recent arts only focus on a particular type, e.g., the noise of measurement errors. 